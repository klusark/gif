% !TEX TS-program = pdflatex
% !TEX encoding = UTF-8 Unicode

% This is a simple template for a LaTeX document using the "article" class.
% See "book", "report", "letter" for other types of document.

\documentclass[11pt]{article} % use larger type; default would be 10pt

\usepackage[utf8]{inputenc} % set input encoding (not needed with XeLaTeX)

%%% Examples of Article customizations
% These packages are optional, depending whether you want the features they provide.
% See the LaTeX Companion or other references for full information.

%%% PAGE DIMENSIONS
\usepackage{geometry} % to change the page dimensions
\geometry{letterpaper} % or letterpaper (US) or a5paper or....
% \geometry{margin=2in} % for example, change the margins to 2 inches all round
% \geometry{landscape} % set up the page for landscape
%   read geometry.pdf for detailed page layout information

\usepackage{graphicx} % support the \includegraphics command and options

% \usepackage[parfill]{parskip} % Activate to begin paragraphs with an empty line rather than an indent

%%% PACKAGES
\usepackage{booktabs} % for much better looking tables
\usepackage{array} % for better arrays (eg matrices) in maths
\usepackage{paralist} % very flexible & customisable lists (eg. enumerate/itemize, etc.)
\usepackage{verbatim} % adds environment for commenting out blocks of text & for better verbatim
\usepackage{subfig} % make it possible to include more than one captioned figure/table in a single float
% These packages are all incorporated in the memoir class to one degree or another...

%%% HEADERS & FOOTERS
\usepackage{fancyhdr} % This should be set AFTER setting up the page geometry
\pagestyle{fancy} % options: empty , plain , fancy
\renewcommand{\headrulewidth}{0pt} % customise the layout...
\lhead{}\chead{}\rhead{}
\lfoot{}\cfoot{\thepage}\rfoot{}

%%% SECTION TITLE APPEARANCE
\usepackage{sectsty}
\allsectionsfont{\sffamily\mdseries\upshape} % (See the fntguide.pdf for font help)
% (This matches ConTeXt defaults)

%%% ToC (table of contents) APPEARANCE
\usepackage[nottoc,notlof,notlot]{tocbibind} % Put the bibliography in the ToC
\usepackage[titles,subfigure]{tocloft} % Alter the style of the Table of Contents
\renewcommand{\cftsecfont}{\rmfamily\mdseries\upshape}
\renewcommand{\cftsecpagefont}{\rmfamily\mdseries\upshape} % No bold!

%%% END Article customizations

%%% The "real" document content comes below...

\title{CMPT 365 Final Project: A GIF encoder}
\author{Joel Teichroeb}
%\date{} % Activate to display a given date or no date (if empty),
         % otherwise the current date is printed 

\begin{document}
\maketitle

\section{A GIF Image}
One of the major reasons GIF was created was that when the internet was just starting to become more widespread in the late 80s there was no good way to represent images. The formats at the time either did not have compression, had poor compression, or were not free for anyone to use. GIF almost solved all of those issues at the time. The major problem with GIF at the time it was released was that the LZW compression algorithm was patented and many people did not want to use GIF images for fear of a law suit. These fears have been mitigated in recent years by the fact that the patents in question are now old enough that they can no longer be enforced.
\subsection{Version}
There are two different versions of GIF, GIF87a and GIF89a. They are overall very similar in that they represent the images in the exact same way, but the more recent GIF89a adds a number of extensions on top of GIF87a that make it possible to do animation.
\subsection{Compression}
The GIF specification combines two different forms of compression together in order to make a fairly good compression scheme that is not overly compute or memory intensive.
\subsubsection{LZW}
GIF employs the LZW compression scheme in order to greatly reduce the size of the image. Because of the patent issues mentioned earlier the LZW compression algorithm was not included in the specification. Instead it just linked to other places where the specification could be found. This made it considerably harder to implement for me as I could not get a really good description of how exactly it works in the case of GIF. LZW uses a code table that stores the different codes it could output. This table has a maximum size of 4096 entries. On initialization of that table it gets filled with the total number of different colour codes that the image can use such that the code at 0 corresponds to the first colour of the image. After the initial colour codes comes the reset code and the stop code. The reset code is used to tell the decoder to reset it's table back to what it started with and it signal the start of the data. The stop code is used to tell the decoder to stop reading. The reset code may be used at any time in order to start a new table. A smart GIF encoder could detect large portions of image that would compress better than others and may want to reset early. A naive encoder would just reset when the table fills up.
\subsubsection{The Data Stream}
Because the codes outputted from LZW are not always going to be 8 bits it becomes necessary to have a way to output numbers smaller or larger than one byte. A bad way to do this would be to just output every number smaller than 8 bits as an 8 bit integer and any number larger than 8 bits as a series of 8 bit integers. Fortunately GIF did not decide to output in such a fashion.

An example from the GIF89a spec:
\begin{verbatim}
     +---------------+
  0  |               |    bbbaaaaa
     +---------------+
  1  |               |    dcccccbb
     +---------------+
  2  |               |    eeeedddd
     +---------------+
  3  |               |    ggfffffe
     +---------------+
  4  |               |    hhhhhggg
     +---------------+
           . . .
     +---------------+
  N  |               |
     +---------------+
\end{verbatim}


In this example eight 5 bit codes are being output from the LZW compression scheme. If it were the case that these needed an entire byte each then an extra 3 bytes would have been used. Over the long run this output mechanism saves quite a bit of space and efficiently allows for variable length codes.

\subsection{Colour Tables}
GIF uses a colour lookup table in order to provide colour to images. There is a global colour table for the entire image, but each sub image can optionally include it's own colour table which would override the global colour table. The use of colour tables has a number of benefits over just encoding the colours straight into the image, but also a few drawbacks. The main advantage is the size of the image. Let's say that you have an image with only eight colours, if a colour table were not used you would have to include all the information for each colour every time it is used. With a colour table all you need to do is provide a number between 0 and 7. This has the added advantage that once compressed each pixel could use at most 3 bits with this table and using the data stream described earlier it will use very little space. The major disadvantage with using a colour table though, in the case of GIF, is that there is a maximum of 255 colours per table. This could mean that there are 255 colours for the entire image, if the encoder is not very good, or it could use a number of local colour tables for sub images, but that would be rather inefficient compared to just using a more modern format like PNG.
\subsection{Multiple Images}

\subsection{Animation}

\section{The Code}
I implemented GIF89a
\subsection{Image Loading}
I needed some way of getting image data into the encoder. I obviously could not use GIF as it would then be harder to tell if I had actually written all the GIF code myself or just made calls to whatever library it was I was using to read the files. For this I used PNG.
I also needed to be able to get multiple frames of data in order to do animation. For that I simply use the command line parameters to specify what each frame should be. If there is more than one frame I include animation extensions in the output GIF file.
\subsubsection{PNG}
The PNG image format is what I chose to read from to get my data. I made sure to only have one function that deals with the actual PNG function calls in order to properly separate the image loading making it so that all the information my encoder got from the image was the pixel data, width, and height. Because I am not an expert in using the PNG library there may be issues with loading certain types of PNG files, such as ones that use an alpha channel, but it works well enough to demonstrate my code.
\subsection{Binary Output}
Because GIF is a binary format I needed a simple way to output binary data. I could have just used the c file api every time I wanted to write to a file, but that would have made the code a lot harder to read. Instead I wrote a simple abstraction around the c file api which just has support for exactly what is needed by the GIF format.
\subsection{GIF}

\section{Issues}
Because of the limited amount of time given for this assignment there are, unfortunately, still a few small outstanding issues with my code.
\subsection{Speed}
One of the main issues that my code has is the LZW compression related to the table search code is slower than it should be. There is certainly room to optimize this code if I had more time on this project. For a 1000 pixel by 1000 pixel image it can take a few seconds to encode on my computer, which is not really acceptable these days. Fortunately though, a smaller image, such as a 100 pixel by 100 pixel image is almost instant to encode.
\subsection{Two Colours}
There is a strange bug that happens when you try to encode an image with only two colours. I think I must have misunderstood something in the GIF89a specification as other tools which output GIF images seem to use a minimum of four colours in their output, even on an image with only two colours.


\end{document}
